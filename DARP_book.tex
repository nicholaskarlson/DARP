\documentclass{book}
\usepackage[utf8]{inputenc}
\usepackage{hyperref}
\usepackage{geometry}
\usepackage[backend=biber, style=authoryear, natbib=true]{biblatex}
\addbibresource{references.bib}

\begin{document}

\title{Data Analysis with R and Python (DARP) \\ \large Including Sentiment Analysis}
\author{}
\date{}
\maketitle

\begin{center}
    MIT License \\
    Available on GitHub at: \\
    \url{https://GitHub.com/nicholaskarlson/DARP}
\end{center}

\tableofcontents

\chapter*{Preface}
In the era of information, data is the cornerstone of transformative change. With every click, like, share, and purchase, vast volumes of data are generated. When used right, this data offers insights that have the potential to drive innovation, streamline processes, and reshape the way we live, work, and think. This book, "Data Analysis with R and Python - Including Sentiment Analysis," aims to provide you with the tools and knowledge to harness the power of data effectively and responsibly. Note that this book has very few references. The reader is encouraged to use resources available on the Web to fact-check. This book's view on ``causation'' and facts is heavily influenced by Mosteller and Tukey \parencite{mosteller1977}.

The world of data analysis is vast, intricate, and continually evolving. This book doesn't aim to cover everything—no single book could. Instead, our purpose is to offer a robust foundation in data analysis using two of the most powerful and versatile languages in the field: R and Python. Both languages offer their own strengths and capabilities, and combined, they form a formidable toolkit for any budding data analyst or scientist.

Our choice of incorporating Google Colaboratory is deliberate. Google Colab, a cloud-based platform, makes data analysis accessible to a broader audience. Whether you're a student, a professional looking to transition into the world of data, or even a seasoned analyst looking to upgrade your skills, Google Colab offers a platform where you can experiment, learn, and implement without worrying about extensive setups or costs.

The inclusion of Sentiment Analysis, a specialized field of study within data analysis, speaks to the world's increasing reliance on textual data. Every day, countless reviews, comments, and feedback flood the digital space. Extracting valuable insights from this vast sea of text is essential, and sentiment analysis stands at the forefront of this endeavor.

The real-world applications we dive into serve a dual purpose. They reinforce the techniques and methods discussed while also illustrating the tangible impacts data analysis can have across various industries. From predicting loan defaults to gauging the sentiment of a social media post, the case studies are a testament to the breadth and depth of data analysis.

This book, in essence, is a journey. It is a journey that begins with understanding the basics of data and culminates in the application of advanced machine learning techniques to real-world scenarios. As with all journeys, there will be challenges along the way. But with perseverance, curiosity, and the foundational knowledge this book provides, you'll be well-equipped to navigate the dynamic world of data analysis.

By the end of this journey, our hope is that you don't just see data as mere numbers or text but as stories waiting to be told, patterns waiting to be discovered, and solutions waiting to be found.

Welcome to "Data Analysis with R and Python." Let the exploration begin.

\chapter{Introduction}
The realm of data analysis serves as a bridge, transforming raw data into actionable insights and understanding. This chapter provides a foundational overview, guiding you through the importance of data analysis, the rationale behind the choice of tools used in this book, and the significance of a cloud-based platform like Google Colaboratory. Additionally, we'll delve into the critical concept of reproducible analysis—a cornerstone in ensuring the integrity and validity of analytical work.

\section{Background on Data Analysis}
Data has often been labeled the 'oil' of the modern era—a driving force behind decisions, innovations, and understanding in numerous fields. This section delves into the evolution of data analysis, tracing its historical roots, its current applications, and its profound impact on both industries and our daily lives.

\section{Why R and Python?}
In the vast landscape of programming languages and tools, R and Python have emerged as premier choices for data analysis. But what makes them so special? Here, we will explore the unique features, strengths, and community support that have elevated these languages to their current esteemed status in the data science community.

\section{Google Colaboratory: A Cloud-Based Data Analysis Platform}
The digital age has seen a shift from local computational setups to cloud-based platforms. This section introduces you to Google Colaboratory, elucidating its advantages, its integration capabilities with R and Python, and its significance in making data analysis more accessible and collaborative.

\section{Reproducible Analysis with Notebooks}
Transparency and reproducibility are fundamental pillars in data analysis. In this segment, we will discuss the importance of ensuring that analytical processes can be consistently replicated, verified, and understood by others. We will introduce the concept of notebooks as tools that champion this cause, fostering a culture of openness and accountability.

\chapter{Setting Up Your Environment with Google Colaboratory}
While understanding theories and concepts is crucial, the practical application is where true learning and discovery happen. This chapter is dedicated to helping you set up a hands-on environment where you can experiment, analyze, and learn. We will walk you through the intricacies of Google Colaboratory, ensuring you're well-equipped to start your analytical journey.

\section{Introduction to Google Colaboratory (Colab)}
A key component of effective learning in data analysis is having the right environment—a place where you can code, visualize, and interpret without hindrance. This section provides an introduction to Google Colaboratory, often referred to as Colab, a platform that offers all this and more, all within the comfort of your web browser.

\section{Google Colab using R or Python}
While Colab is intrinsically linked with Python, it is versatile enough to support other languages, notably R. In this segment, we will guide you through the process of setting up both Python and R environments within Colab, ensuring you have the flexibility to work with either, depending on the task at hand.

\section{The Basics of Google Colab}
Before diving deep into data analysis, it's essential to familiarize yourself with the platform you'll be using. This section is a hands-on guide, walking you through the basic functionalities, features, and tips to make the most of Google Colaboratory.

\chapter{Foundations of Data Analysis}
Before diving into the intricate methodologies and algorithms associated with data analysis, it's essential to understand the foundational principles that form the backbone of any analytical venture. This chapter is designed to provide a solid base, ensuring that you're well-versed in the types of data you might encounter, the environment you'll be working in, and the fundamental distinction between different data collection methodologies that can have a profound impact on the conclusions you draw.

\section{Data Types and Structures}
Data, in its many forms, is the raw material for analysis. Grasping the various data types and structures is akin to a craftsman knowing their tools. This section delves deep into the diverse types of data you might encounter, from quantitative to categorical, and the structures that house them, ensuring you can effectively handle and manipulate data for analysis.

\section{Working with Data in Colab Notebooks}
Google Colaboratory is not just a coding platform—it's an environment where data comes to life. Here, we'll explore the nitty-gritty of importing, visualizing, and manipulating data within Colab notebooks, ensuring seamless transitions between data processing steps.

\section{Observational versus Experimental Data and Causality}
Data doesn't exist in a vacuum. The way it's collected and the context it stems from play crucial roles in determining its analytical value. In this section, we will demystify the distinctions between observational and experimental data, leading into a deeper understanding of causality, correlation, and the stories data can—and cannot—tell.

\chapter{Exploratory Data Analysis (EDA)}
Embarking on the analytical journey requires a map, a preliminary understanding of the data terrain you're about to traverse. Exploratory Data Analysis, or EDA, is that map. This chapter sheds light on the initial steps of any data analysis endeavor, focusing on summarizing, visualizing, and understanding the data before more complex analytical methods are employed.

\section{Descriptive Statistics}
The heartbeat of any dataset lies in its central tendencies, variations, and distributions. This section introduces you to the world of descriptive statistics, where numbers tell tales of averages, spreads, and patterns, offering a summarized view of the data in your hands.

\section{Data Graphing Basics}
A picture, they say, is worth a thousand words. In the realm of data, a well-crafted graph can communicate complexities that numbers might struggle with. This segment offers a primer on fundamental data visualization techniques, ensuring you can paint accurate, insightful pictures with your data.

\chapter{Visualization Techniques}
Moving beyond the basics, data visualization is both an art and a science. It's about making informed choices to present data in ways that are both insightful and intuitive. This chapter dives deeper into advanced graphing techniques, differentiated by the analytical powerhouses of R and Python, offering tools and techniques to bring data to life.

\section{Graphing Data with Python}
Python, with its diverse libraries and extensive community support, is a haven for data visualization. In this section, we'll explore the plethora of graphing options Python offers, from simple bar charts to intricate heat maps, ensuring you can choose the right visualization for your data story.

\section{Graphing Data with R}
R, often considered the statistician's best friend, boasts a rich lineage of data visualization methods. This segment delves into the world of R graphing, highlighting the unique features and capabilities that make R a favored choice for many data analysts and researchers.

\chapter{Statistical Analysis}
The crux of data analysis often lies in deciphering patterns, making predictions, and drawing conclusions based on evidence within the data. This is the realm of statistical analysis. This chapter delves into foundational statistical methods, each acting as a powerful tool to extract meaning from data. Whether you're testing a theory, identifying relationships, or categorizing data, understanding these techniques is paramount.

\section{Hypothesis Testing}
At the core of many scientific endeavors is a question—a hypothesis waiting to be validated or refuted. This section introduces you to the structured world of hypothesis testing, where data is used to make informed decisions about underlying population parameters, helping you discern signal from noise.

\section{Correlation and Regression Analysis}
Relationships are at the heart of understanding complex systems, and data is no exception. Here, we'll delve into how variables interact and influence one another, exploring both the strength and nature of these relationships. By the end, you'll be equipped to predict, understand, and visualize interactions within your data.

\section{Analysis of Variance (ANOVA)}
Differences abound in the world around us, and data is no stranger to this. ANOVA is a powerful technique designed to test if there are significant differences between means of multiple groups. This section will guide you through its principles, applications, and intricacies.

\section{Multivariate Statistical Analysis}
Multivariate Statistical Analysis refers to the procedures for analysis of data that involve more than one variable at a time. It is essential in the field where multiple phenomena are observed simultaneously to understand the relationships between them and the outcomes they produce. This area of statistics encompasses many models, hypothesis tests, and exploratory techniques.

This section introduces several key multivariate techniques such as factor analysis, which is often employed to identify the underlying structure in data; and cluster analysis, which is used to classify objects or cases into relative groups called clusters.

We will also touch upon Multivariate Analysis of Variance (MANOVA), which extends the ANOVA framework to accommodate multiple continuous dependent variables, and Canonical Correlation Analysis (CCA), which explores the relationships between two sets of variables. Through examples and practical exercises, you will learn how these techniques can uncover more complex structures in data and lead to more profound insights.

There is a lot of material to cover in Multivariate Statistical Analysis, so it will be given its own chapter found later in this book.

\section{Logistic Regression}
While linear regression is a staple in predicting continuous outcomes, logistic regression steps in when the outcome is binary. From predicting customer churn to diagnosing medical conditions, this section will take you through the fundamentals of logistic regression, a staple in the statistical toolkit.

\chapter{Multivariate Statistical Analysis}
\label{chap:multivariate_statistical_analysis}

Multivariate Statistical Analysis encompasses a suite of techniques essential for understanding and interpreting the complexity inherent in data involving multiple variables. Unlike univariate analysis, which considers one variable at a time, or bivariate analysis, which examines the relationship between two variables, multivariate analysis deals with the simultaneous observation and analysis of more than two statistical outcome variables. This approach is pivotal in scenarios where variables are interrelated and where the effect of such variables needs to be separated and understood.

\section{Sorting and Grouping}
Multivariate methods such as cluster analysis help in sorting and grouping data by identifying structures or clusters within the data. By finding clusters of data that are similar to each other, these techniques provide insights into the natural groupings within the data, which can be crucial for market segmentation, image recognition, and other applications in machine learning.

\section{Sorting and Grouping: An Example}
To illustrate sorting and grouping, consider the following simple dataset that includes measurements from three different variables: `Age`, `Income`, and `Score`. These variables could represent customer data for a company, where `Age` and `Income` are self-explanatory, and `Score` could be some kind of engagement or satisfaction metric.

\begin{table}[h]
\centering
\begin{tabular}{c c c}
\hline
\textbf{Age} & \textbf{Income (K)} & \textbf{Score} \\
\hline
25 & 50 & 75 \\
40 & 65 & 60 \\
30 & 55 & 80 \\
45 & 80 & 70 \\
35 & 60 & 90 \\
\hline
\end{tabular}
\caption{Sample data for sorting and grouping}
\label{tab:sample_data}
\end{table}

In multivariate analysis, we would use cluster analysis techniques to sort and group this data. For example, a simple k-means clustering might identify groups based on age and income, or income and score, depending on the number of clusters we choose to define. Through this process, we might find that younger customers have a higher score but lower income, indicating a potential target demographic for certain marketing strategies. Similarly, machine learning algorithms can use this clustered information to make predictions or to tailor services to each group's preferences.

\subsection{Demonstrating Sorting and Grouping with Python}
To better understand how sorting and grouping might be accomplished, we can use a Python script executed in a Jupyter Notebook to apply a k-means clustering algorithm to our sample data. K-means clustering will partition the data into $k$ number of clusters where each observation belongs to the cluster with the nearest mean. This is a method to identify groups (clusters) in which data points are similar to each other.

The following Python code performs a k-means clustering with $k=2$, which is a simple starting point for our dataset:

\begin{verbatim}
# Python code for k-means clustering
import numpy as np
from sklearn.cluster import KMeans
import matplotlib.pyplot as plt

# Sample data
data = np.array([
    [25, 50, 75],
    [40, 65, 60],
    [30, 55, 80],
    [45, 80, 70],
    [35, 60, 90]
])

# Apply k-means clustering
kmeans = KMeans(n_clusters=2, random_state=0).fit(data)
labels = kmeans.labels_

# Add cluster labels to our data
clustered_data = np.concatenate((data, labels[:, None]), axis=1)

# Display the data points and their cluster assignments
plt.figure(figsize=(8, 6))
plt.scatter(clustered_data[:,0], clustered_data[:,1], c=clustered_data[:,2], cmap='viridis')
plt.title('Sample Data - K-Means Clustering')
plt.xlabel('Age')
plt.ylabel('Income (K)')
plt.colorbar(label='Cluster Label')
plt.show()
\end{verbatim}

The code first imports necessary libraries, defines the sample data, and applies the k-means algorithm. It then plots the data points, coloring them according to the cluster they belong to. This visual representation helps to intuitively understand how the k-means algorithm has grouped the data based on the age and income of the individuals. It can provide insight into customer segmentation, which is invaluable for targeted marketing, product development, and other business strategies. It also serves as a straightforward example of the power of machine learning for pattern recognition within datasets.

\subsection{Interactive Analysis with Google Colab}
For a hands-on experience, the Python code for k-means clustering is also hosted on Google Colab. Google Colab is a free online platform that offers an environment to write and execute Python. It's particularly useful for machine learning, data analysis, and education.

To interact with the code, tweak parameters, or experiment with different clustering algorithms, access the Jupyter notebook hosted on Google Colab. Simply click on the following link and you will be directed to the notebook in your web browser:

\begin{center}
\href{https://colab.research.google.com/drive/1C4Rnja-fjwVB6GI5h1vYItJ5fbxDHlxh}{\textbf{Open the Sorting and Grouping Notebook}}
\end{center}

Feel free to copy the notebook to your drive, modify it, or even use it as a starting point for more complex analysis. This is an excellent opportunity to explore the practical aspects of data science and machine learning in an easily accessible manner.

\section{Studying Dependence of Variables}
Methods like multiple regression and canonical correlation analysis allow researchers to study the dependence of variables on one another. These dependencies can be linear or nonlinear and may involve interactions and correlations among multiple variables. In machine learning, understanding these dependencies is crucial for feature selection and improving model accuracy.

\section{Structural Simplification or Data Reduction}
Techniques such as principal component analysis and factor analysis aim at structural simplification or data reduction. They reduce the number of variables to a few interpretable linear combinations that capture the most significant information. In machine learning, this dimensionality reduction is vital for simplifying models, reducing overfitting, and improving computational efficiency.

\section{Hypothesis Testing}
Multivariate Analysis of Variance (MANOVA) extends the ANOVA framework to test multiple dependent variables simultaneously, providing a more robust analysis when multiple measurements are taken. Similarly, machine learning often involves complex hypothesis testing when validating models, requiring multivariate techniques to assess model performance across multiple dimensions.

\section{Prediction}
Prediction is the cornerstone of machine learning, and multivariate methods are at the heart of predictive analytics. Multiple regression, for example, predicts the value of a variable based on the values of others. Machine learning extends this idea to more complex models that can handle vast datasets and predict outcomes with high accuracy.

In summary, multivariate statistical analysis methods not only provide tools for effective analysis of complex datasets but also form the foundation of many machine learning algorithms. The methods discussed are not just abstract statistical concepts but practical tools that drive the decision-making process in business, technology, and science.


\chapter{Machine Learning Foundations}
In a world inundated with data, the ability to teach machines to learn from this data is no longer a luxury—it's a necessity. Machine learning is the frontier where computers derive insights, make predictions, and even undertake decisions based on patterns in data. This chapter serves as a foundation, introducing you to the core concepts, types, and evaluation metrics of machine learning models.

\section{Introduction to Machine Learning}
Stepping into the world of machine learning can seem daunting, but at its core, it's about understanding and leveraging patterns. This section offers an overarching view of what machine learning entails, its applications, and its transformative potential in numerous domains.

\section{Supervised vs. Unsupervised Learning}
Machine learning is a diverse field, with models tailored to different tasks and data types. This segment dives into the primary categories of machine learning—supervised and unsupervised learning—exploring their nuances, applications, and distinguishing features.

\section{Model Evaluation and Validation}
Building a machine learning model is only part of the journey. Ensuring it performs well, is robust, and generalizes to unseen data is crucial. Here, we'll delve into the techniques and metrics to evaluate and validate your models, ensuring they're not just fit but are truly in shape for the tasks at hand.

\chapter{Machine Learning with Python}
Python has rapidly ascended to become a premier language in the data science and machine learning community, thanks to its versatility, readability, and a rich ecosystem of libraries. One such library, Scikit-learn, stands as a pillar for machine learning in Python. This chapter dives into the application of various machine learning models using Python, providing practical insights and guiding you to harness Python's power for analytical endeavors.

\section{Regression Models with Scikit-learn}
Predicting continuous values is a common task in machine learning, be it forecasting sales, estimating house prices, or predicting stock movements. This section introduces you to regression models using Scikit-learn, providing a hands-on guide to building, tuning, and evaluating predictive models that can capture and leverage relationships within your data.

\section{Classification Models with Scikit-learn}
Beyond predicting numbers, there's often a need to categorize data, whether it's identifying spam emails, diagnosing diseases, or classifying images. Here, we explore classification models with Scikit-learn, diving into techniques that can discern and predict categorical outcomes based on input data.

\section{Clustering and Dimensionality Reduction}
Data, in its raw form, can be vast and intricate. Sometimes, the insights lie not in individual data points but in the clusters they form. At other times, reducing the dimensions can unveil patterns obscured by the data's complexity. This section explores clustering techniques and dimensionality reduction methods, offering tools to simplify, group, and extract insights from vast datasets.

\chapter{Machine Learning with R}
R, traditionally a language for statisticians, has embraced the rise of machine learning with open arms. Its comprehensive packages and robust modeling capabilities make it a formidable tool for data analysts and researchers. This chapter delves into machine learning with R, guiding you through the intricacies of building, evaluating, and refining models in this powerful language.

\section{Regression Models with R}
While Python has its strengths, R brings to the table a rich statistical heritage, making it particularly adept at tasks like regression. This section focuses on building regression models with R, offering insights into leveraging R's statistical foundations for predictive analytics.

\section{Classification Models with R}
R's extensive suite of packages and functions makes it a versatile tool for classification tasks. From logistic regression to support vector machines, this segment introduces you to a range of classification techniques in R, ensuring you can make informed decisions about categorizing data.

\section{Decision Trees and Random Forests in R}
Decision Trees offer a visual and intuitive way to make decisions based on data, while Random Forests enhance this by adding an ensemble approach. In this section, we will dive deep into these tree-based methods in R, exploring their capabilities, strengths, and applications in various domains.

\chapter{Deep Learning Basics}
In the ever-evolving landscape of machine learning, deep learning stands as a revolutionary leap forward. Mimicking the workings of the human brain through intricate networks of artificial neurons, deep learning techniques have shown unparalleled success in tasks ranging from image and speech recognition to natural language processing. This chapter will introduce you to the core concepts of deep learning, guiding you through its foundational elements and its implementation in Python.

\section{Introduction to Neural Networks}
At the heart of deep learning lies the neural network, a computational model inspired by the intricate networks of neurons in the human brain. This section provides an overview of these networks, from their foundational principles to their diverse architectures, setting the stage for the more advanced techniques that follow.

\section{Building Deep Learning Models with TensorFlow in Python}
TensorFlow, developed by Google, stands as one of the premier libraries for deep learning. Its flexibility, scalability, and extensive capabilities make it a favorite among both beginners and experts. This segment delves into constructing deep learning models using TensorFlow, offering insights into the library's vast features and guiding you through hands-on implementations.

\section{Deep Learning in Python using Keras}
While TensorFlow is powerful, its complexity can be daunting. Enter Keras—a high-level neural networks API that runs on top of TensorFlow, simplifying and streamlining the process of building and training deep learning models. In this section, we'll explore how to harness the power of Keras, ensuring you can build, train, and evaluate models with ease.

\chapter{Sentiment Analysis}
The digital age has ushered in an explosion of textual data—reviews, tweets, comments, and more. Within these words lie sentiments, emotions, and opinions waiting to be understood. Sentiment analysis stands at the crossroads of machine learning and natural language processing, dedicated to extracting, understanding, and interpreting these sentiments. This chapter takes you on a journey through the intricacies of sentiment analysis, from its foundational concepts to advanced modeling techniques.

\section{Introduction to Sentiment Analysis}
Every piece of text, from a simple tweet to a lengthy review, carries with it an emotion, an opinion, a sentiment. Understanding this sentiment can offer businesses, researchers, and individuals valuable insights. This section introduces you to the world of sentiment analysis, elucidating its importance, applications, and the challenges it poses.

\section{Text Preprocessing and Feature Extraction}
Text, in its raw form, is messy. For machines to understand and analyze it, this text needs to be processed and transformed into a structured format. Here, we'll delve into the crucial steps of text preprocessing, from tokenization to stemming, and explore techniques to extract meaningful features from the processed text.

\section{Sentiment Classification Models}
Once text has been processed and features have been extracted, the stage is set for modeling. This segment introduces various machine learning and deep learning models tailored for sentiment classification, guiding you through their construction, training, and evaluation.

\section{Interpretation of Sentiment Data}
While predicting sentiment is crucial, understanding why a model made a specific prediction can be equally valuable. This section dives into the interpretation of sentiment data, offering techniques and insights to unveil the underlying reasons for a model's predictions, ensuring transparency and trustworthiness in your analyses.

\chapter{Case Studies: Real-World Applications}
Theory and practice often walk hand in hand. While understanding the theoretical intricacies of data analysis is fundamental, seeing these concepts in action offers unparalleled insights. This chapter delves into real-world applications, each case study acting as a window into the practical challenges, intricacies, and transformative potential of data analysis in diverse domains. Through these studies, you'll witness the tangible impact of the techniques and methodologies explored in previous chapters.

\section{Customer Segmentation and Churn Prediction}
Every customer has a unique journey, a distinct pattern of interaction with a business. Understanding these patterns can be pivotal for businesses aiming to optimize customer experiences, loyalty, and value. This case study explores the art and science of segmenting customers based on their behaviors and predicting potential churn, offering actionable insights for businesses to enhance retention and growth.

\section{Loan Default Prediction}
Financial institutions hinge on the delicate balance of risk and reward. One of the most significant risks they face is the potential for a borrower to default on a loan. This case study delves into the world of financial analytics, where data analysis techniques are employed to predict the likelihood of loan defaults, aiding institutions in making informed lending decisions.

\section{Sentiment Analysis in Social Media}
The digital age has transformed social media platforms into global stages where opinions, emotions, and sentiments are freely expressed. For businesses, policymakers, and researchers, these platforms offer a goldmine of insights. This case study dives deep into the realm of social media, exploring how sentiment analysis techniques can extract, analyze, and interpret the emotions and opinions voiced online, providing a pulse of the global sentiment on various topics.

\chapter*{Conclusion and Further Reading}
Every journey, no matter how enlightening, must find its conclusion. As we wrap up our exploration into data analysis with R and Python, this section reflects on the journey undertaken, the lessons learned, and the vistas opened. However, the world of data is vast and ever-evolving. While this book aimed to offer a comprehensive foundation, there are countless avenues yet to explore. Here, we'll guide you on potential next steps, offering recommendations for further reading, ensuring your analytical journey continues to flourish.

The difference between experimental data and observational data is imporant. There are semester-long courses in statistics regarding the collection and analysis of data from controlled experiments. That said, most of the data that data analysts analyze in the era of "big data" is observational data. Forecasting with the use of observational data can be quite accurate however, so using observational data to make decisions can be very effective. However, the issue of causality is more easily addressed with experimental data. Two giants in the field of statisics discuss causality in their book, "Data Analysis and Regression: A Second Course in Statistics." If you are interested in causality this book is worth a look. The book's reference is: Mosteller, F., and Tukey, J. W. (1977). Data Analysis and Regression: A Second Course in Statistics. Reading, MA: Addison-Wesley Pub Co.

\printbibliography
\end{document}










